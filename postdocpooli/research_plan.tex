% Use document class "aletter.cls" (for letters):
\documentclass[11pt, oneside]{essay}

% ISO 8859-1 character encoding is assumed
\usepackage[finnish, english]{babel}
\usepackage[latin1]{inputenc}
\usepackage[T1]{fontenc}
\usepackage{natbib}
\usepackage{color}
\usepackage{amsmath, amssymb}

\newcommand{\tred}[1]{\textcolor{red}{#1}}

\title{Research Plan}
\author{KyungHyun Cho}

\begin{document}

\maketitle

\section{Background: Deep Learning for Artificial
Intelligence}

Deep learning has gained its popularity recently as a way of
learning complex and large probabilistic models \citep[see,
e.g.,][]{Bengio2009a}.  Especially, deep neural networks
such as a deep belief network (DBN) \citep{Hinton2006} and a
deep Boltzmann machine (DBM) \citep{Salakhutdinov2009a} have
been applied to various machine learning tasks with
impressive improvements over conventional approaches such as
support vector machines (SVM) and \emph{k}-nearest neighbors
(kNN).  Examples include, but are not limited to,  
handwritten digit classification
\citep{Hinton2006,Salakhutdinov2009a}, semantic hashing
for document clustering
\citep{Hinton2006,Salakhutdinov2009s}, natural image
classification
\citep{Krizhevsky2009,Krizhevsky2010,Coates2010,Ranzato2010,Ranzato2010a}, and 
speech/phone recognition \citep{Mohamed2010}.

A learning algorithm for Boltzmann machine (BM) and its variants
has been introduced already in 1985 by \citet{Ackley1985}.
However, training Boltzmann machines was considered to be
difficult due to its stochastic nature and the computational
difficulty until recently. Although contrastive divergence
(CD) learning proposed by \citet{Hinton2002} was considered
a break-through that enabled learning a special-case of BM,
called restricted Boltzmann machines (RBM), it is still
generally accepted that learning even RBM is a difficult
task \citep[see, e.g.,][for examples of poor training
results of RBMs]{Schulz2010,Fischer2010,Desjardins2009}.

%Despite the difficulty in learning, more and more
%publications are made each year and introduce better
%learning methods and different variants of deep models that
%out-perform many other conventional machine learning models.

Improving the learning algorithms for BMs and RBMs and
proposing better deep models have been
an area of active research in the recent years, evidenced by
the increasing number of scientific publications.
For instance, \citet{Lee2009} proposed building a
convolutional deep belief network (cDBN) with a generative
probabilistic pooling mechanism for hierarchical sparse
representation of data. More recently, the use of
rectified linear units instead of binary hidden neurons has
been proposed by \citet{Nair2010} and was extended to deeper
models such as deep sparse rectifier neural network
\citep{Glorot2011} which is highly inspired by the actual
mechanism of biological neurons. Also, \citet{Courville2011}
proposed adding a slab unit to each hidden neuron to expand
the representational power of RBMs to learn continuous data.
Their experiments show that the proposed spike-and-slab RBMs
outperform conventional forms of RBMs.

\section{Research Plan}

Main causes of the difficulties of training RBM which is a
basic building block for constructing deep generative models
are the bias introduced by the approximate gradient learning
such as CD learning, the difficulty of choosing the right
learning hyper-parameters such as learning rates, and the
lack of invariance of the learning rules to the
representation of data.

Based on an advanced Markov-Chain Monte-Carlo sampling
method called parallel tempering (PT), it is expected that
learning BMs can overcome the problem of the approximate
learning methods. With Gibbs sampling used in the learning
algorithms replaced with PT sampling, BMs can learn better
generative models without false probabilistic modes which
are inevitable when the approximate learning method such as
CD learning is used.

A problem of choosing appropriate learning parameters while
learning is that in most cases it is computationally
infeasible to estimate the objective function of learning.
In other words, it is not possible to determine whether the
trained model by the chosen hyper-parameters is good or not.
It is, however, possible to estimate it with minimal
computational cost while learning by adapting the idea of
simple importance sampling. Based on this estimated cost,
any hyper-parameter including a learning rate can be chosen
on-the-fly.

A closer observation into the update rules of BMs and RBMs
suggested that learning by the conventional update rules is
easily distracted depending on the representation of data
sets. Based on this observation, it is anticipated that a
new set of update rules that takes geometrical property of
the parameter space into account can be designed. It can be
considered a more efficient alternative to using the
second-order statistics directly.

Once these methods are investigated and tested on a simple
RBM with various data sets including MNIST handwritten
digits \citep{Lecun1998} and Caltech 101 Silhouettes
\citep{Marlin2010}. More research on extending the
new learning algorithms to continuous data sets, for
instance NORB stereo-version objection recognition
\citep{Lecun2004} and CIFAR-10 natural image classification
\citep{Krizhevsky2009} must be
conducted.

Applying the learning methods to RBM is only a beginning, as
they must be able to work also for deeper, more complex
models such as DBN and DBM.  The very first step is, then, to
investigate the existing deep models carefully, both 
theoretically and empirically. 

Theoretical analysis requires me to mathematically
investigate the learning and inference algorithms of the
deep models, especially focusing on the conventional
algorithms. On the other hand, I will implement the existing
deep models and test them with various benchmark and real
data sets to confirm what have been found by the theoretical
analysis.  At the end of this stage, it is expected that
 difficulties and their root causes in learning deep models will
be determined.

It is 
expected that the new learning algorithms proposed for RBMs
can be applied to more complex, deeper models with
modifications that address the determined difficulties.
However, there will be many research issues on designing and
modifying learning algorithms due to the high complexity of
the deep architectures.

Furthermore, the existing deep models might turn out to be
inappropriate or sub-optimal for learning wide ranges of
data sets. In this case, new, better deep models must be
proposed that can overcome problems of the existing ones,
together with suitable learning algorithms.

With the newly proposed architectures and learning
algorithms, extensive experiments are required to validate
them. Both carefully crafted synthetic data sets,
benchmark data sets, and real-world data sets are to be
tested on. Especially, the validation using the benchmark
data sets are important, as they indicate the least
usefulness and performance of the proposed models and
algorithms.

Finally, new applications of the proposed methods will be
sought. Deep learning so far has mainly concentrated on
rather simple benchmark problems, and
unfortunately, this limitation of the experimental focus has
hindered researchers from other fields to adopt deep models
in their research. Thus, it is important for me as a
researcher in deep learning to make a sample case for applying
deep learning to other applications such as bioinformatics,
neuroinformatics, and time-series analysis.

%\subsubsection{Expected Research Outcome}
%
%Personally, the doctoral research will result in a number of
%publications both in conferences in journals. Mainly, they
%will cover the theoretical and experimental finding of yet
%unknown properties of deep models, the proposals of new
%models and improved learning algorithms, and the applications
%of the newly proposed models. Also, the newly proposed
%models and learning algorithms will be made publicly
%available and promote the adaptation of deep learning by
%researchers from various fields. 
%
%In the aspect of deep learning, it will potentially replace
%and/or augment many conventional models used in various
%fields. For instance, principal component analysis (PCA)
%which is used frequently for dimensionality reduction can be
%replaced by deep auto-encoder \citep{Hinton2006}. Deep
%models can be used as learning better priors for
%conventional shallow probabilistic models such as
%independent component analysis (ICA) and factor analysis
%(FA). 
%
%Ultimately, the research will enable both academic
%researchers from other fields and professionals from various
%industries to easily use deep learning for their need, such
%as how FastICA \citep{Hyvarinen1999} enabled the wide-use of
%independent component analysis (ICA) in research fields
%ranging from neuroscience, image processing to climate
%studies.

\subsection{Research Schedule}

\subsubsection{First Two Years: Sep. 2011 - May. 2012}

\textbf{Research goals} It is expected that the improved
learning algorithms proposed in the conferences and the
thesis with their application to more general deep models
will be weaved into a journal publication. The paper will
include the improved learning methods such as the parallel
tempering learning \citep{Cho2010}, the enhanced gradient,
the adaptive learning rate \citep{Cho2011a}, and their
extensions to continuous data \citep{Cho2011b} that have
already been reported through conference proceedings as well
as more theoretical explanation and experiments. Also, the
application of the proposed algorithms to deep models will
be included. The possible target journal is the Journal of
Machine Learning Research where many seminal papers on deep
learning have been published. I, with the instructors,
expect to submit the paper by Dec. 2011.

While preparing and submitting the journal paper, I will
conduct more experiments from May 2011 to Dec. 2011 on deep
Boltzmann machines (DBM) and convolutional DBMs in order to
find root causes of difficulties in learning those models
that prevent them from being widely used. Once those causes
are determined, mathematically solid solutions including a
change to the model architecture and the learning algorithms
will be designed. It will lead to a number of papers in
peer-reviewed conferences by the summer of 2012.  The Neural
Information Processing Systems Foundation (NIPS) 2011, the
International Conference on Machine Learning (ICML) 2012,
the International Conference on Artificial Intelligence and
Statistics (AISTATS) 2012 are main target conferences for
reporting the proposed learning algorithms and experimental
results.

Deep learning requires heavy computation that usually takes
from several hours to several weeks. Advanced computing
capability is, thus, of extreme importance. While I will be
conducting the main research on deep learning, it will be
necessary for me to implement the experiment code into GPU
computing and cluster-based super-computers. Implementing
and optimizing the code for those special-purpose computing
systems require extensive workload and time, but it is
certain that research will gain huge momentum from using
them. Once it is implemented and tested internally in the
department and the group, it will be possible to make the
implementation publicly available through the Internet
during the spring of year 2012.

\textbf{Study goals} The doctoral programme requires the
student to complete 60 credits worth courses that are divided
into three categories which are scientific principles,
research field, and supplementary field. I plan to finish
all required courses during the first three academic
semesters (Sep. 2011 - Dec. 2012) in order to build up the
fundamental knowledge in machine learning and dedicate
himself to the research afterward. 

\subsubsection{Later Years: Jun. 2012 - 2015}

\textbf{Jun. 2012 - Nov. 2013} In order to follow recent
research trend in deep learning, I will make two visits to
research institutes abroad. The visits will serve two
purposes; (1) widening my theoretical research by exchanging
ideas on learning algorithms with other researchers, and (2)
learning application tasks in which conventional machine
learning models are not satisfactory. Please, refer to the
attached mobility plan and invitation letters for the
destinations and the detailed descriptions.

\textbf{Dec. 2013 - May. 2014} As deep learning has been
highly successful in computer vision and image analysis
tasks, I will concentrate on specializing various deep
models and learning algorithms for those tasks. It will
require implementing and investigating convolutional deep
neural networks and BMs with more general hidden variables,
for example, spike-and-slab variables.

Along the continuing research real data sets will be
gathered in order to test newly proposed architectures and
learning algorithms. ZenRobotics,
Finland\footnote{http://www.zenrobotics.com/} is a
technology company where I worked for half a year as a
student researcher working on computer vision system in
robotics. It will be possible for me to obtain real data
sets that are actually used in practice to experiment the
deep models and their learning algorithms.

\textbf{Jun. 2014 - 2015} Based on research already done,
the dissertation and journal papers that will report the new
learning algorithms and applications will be prepared. The
topic for the dissertation will be \emph{learning algorithms
and applications of deep probabilistic models}. The
dissertation defense is expected to take place between Dec.
2014 to May. 2015, which will depend on the outcome from
earlier years of the doctoral research.

Future research direction will be also sought meanwhile in
order to further my research continuously after doctoral
degree.

\section{Financial plan}

Scholarship from Finnish Doctoral Programme of Computational Science
(FICS) will be the top priority for funding my doctoral
study, as it does not only cover the basic living expense,
but possible research activities including the support for
attending conferences and visiting research institutes
abroad. As my doctoral research requires heavy collaboration
with world-renowned institutes abroad, FICS is an ideal
graduate school that can financially and academically
support me. 

Additionally to living and research expense from FICS, it is
expected for me to apply for extra funding from other
organizations such as Wihuri foundations for advanced
computing equipments and other research-related expense,
such as workstations equipped with latest graphical
processing units that can significantly reduce the time
required for running experiments. Also, I am ready for any
funding possibilities that may help increase research
efficiency.

\small
\bibliographystyle{research_plan}
\bibliography{research_plan}





\end{document}
