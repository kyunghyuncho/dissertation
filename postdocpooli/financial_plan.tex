% Use document class "aletter.cls" (for letters):
\documentclass[11pt, oneside]{essay}

% ISO 8859-1 character encoding is assumed
\usepackage[finnish, english]{babel}
\usepackage[latin1]{inputenc}
\usepackage[T1]{fontenc}
\usepackage{natbib}
\usepackage{color}
\usepackage{amsmath, amssymb}

\newcommand{\tred}[1]{\textcolor{red}{#1}}

\title{Financial Plan}
\author{KyungHyun Cho}

\begin{document}

\maketitle

\section{Cost Summary}



\section{Cost Break-Down}

\subsection{Research}

\subsubsection{Equipments}


One important reason for choosing the Machine Learning Lab at the
University of Montreal as a post-doctoral research destination is
the readily availability of a large amount of computing power. At
the Machine Learning Lab I can access computing resources from
more than three distinct clusters; a nation-wide cluster,
multiple state-wide clusters and the lab-specific cluster.

Those clusters are, however, mainly focused on running a massive
number of parallel jobs, which is more suitable for
systematically exploring and searching the optimal
hyperparameters and model structures. For more preliminary
experiments on understanding how models behave and their learning
dynamics are, it is important to have a dedicated, personal
high-performance computing machine.  Furthermore, the machine
needs to be portable so that research continues without
interruption even during conference trips as well as possible
research visits to international collaborators (See the next
section for the details on research visits).

MacBook Pro from Apple is one laptop that satisfies both
performance and portability. The laptop costs \textbf{1649 EUR}
(with the size of memory upgraded to 16G).



\subsubsection{Research Visit}





\subsubsection{Travels}

It is crucial for me to attend top-class international
conferences regularly to keep up with the latest trend in deep
learning as well as natural language processing research.
Furthermore, those international conferences are the places where
I will present my own latest research findings to other
world-class researchers, which will help me as a researcher from
Finland gain recognition.

Each conference attendance costs approximately \textbf{2500 EUR}.
The estimate is based on the conferences I attended in 2012 and
2013. The cost for each travel may further be broken down into:
\begin{enumerate}
    \item Flight: approx. 1200 EUR (intercontinental)
    \item Transportation: approx. 100 EUR
    \item Accomodation: approx. 600 EUR (6 nights)
    \item Registration Fee: approx. 600 EUR
\end{enumerate}

There are \textbf{four} main conferences in deep learning every
year that I must attend to keep up with latest research findings
from top-class scholars and to present my work to them. They are:
\begin{enumerate}
    \item International Conference on Machine Learning (ICML)
    \item Advances in Neural Information Processing Systems (NIPS)
    \item International Conference on Learning Representation (ICLR)
    \item International Conference on AI \& Statistics (AISTATS)
\end{enumerate}

Additionally, as I will conduct research on applying advanced
deep learning algorithms and models to natural language
processing, it is important for me to attend at least
\textbf{two} out of the following
four conferences on statistical natural language processing:
\begin{enumerate}
    \item The Annual Meeting of the Association for Computational Linguistics (ACL)
    \item The Conference of the European Chapter of the
    Association for Computational Linguistics (EACL)
    \item International Conference on Computational Linguistics (COLING)
    \item Empirical Methods in Natural Language Processing (EMNLP)
\end{enumerate}

In total, each year I will need travel budget of at least $2500
\times 6 = \mathbf{15000}$ \textbf{EUR}.


\subsection{Moving}

\subsubsection{Immigration: Work Permit Processing Fee}

It requires a work permit to enter Canada as a post-doctoral
researcher. The processing fee is \textbf{150 CAD}, and
there will be additional cost such as postal fee (A
resident in Finland must apply for Canadian visa through the
Canadian consular in London, UK). Therefore, I would like to
request \textbf{300 EUR}.

\subsubsection{Flight}

It costs approximately \textbf{1200 EUR} (Finnair, British
Airway) for a round-trip flight between Helsinki, Finland
and Montreal, QC, Canada. As it is more expensive to book an
one-way flight (at least, 1600 EUR), I will buy a round-trip
ticket. 

\subsubsection{Parcels}

As I have already lived in Finland for more than four years,
I will need to bring more than what is allowed on the flight
(often, two 23-KG bags). Those excess luggages will be sent
using the \textit{Postal Parcel International} service
provided by Posti. Each 30-KG box costs \textbf{136 EUR},
and I am planning to send \textbf{two} boxes, totaling
\textbf{272 EUR}.

\subsubsection{Temporary Accommodation}

There is high uncertainty on how well the date of my arrival and
the availability of housing will align. I expect to stay at most
\textbf{two weeks} in a temporary accommodation. Although it may
turn out to require much lower, I would like to reserve at least
\textbf{1200 EUR} for the temporary accomodation.


\subsection{Living}

\subsubsection{Salary}

The monthly salary will be covered by my host, Prof. Yoshua Bengio.

\subsubsection{Housing}

Monthly rental fee for a suitable studio apartment near the
University of Montreal is approximately \textbf{680 EUR}. For
each year (\textbf{12} months), I would like to request for
\textbf{8160 EUR} each year for housing.

\subsubsection{Other Living Cost}








%\small
%\bibliographystyle{research_plan}
%\bibliography{research_plan}





\end{document}
