% Use document class "aletter.cls" (for letters):
\documentclass[11pt, oneside]{essay}

% ISO 8859-1 character encoding is assumed
\usepackage[finnish, english]{babel}
\usepackage[latin1]{inputenc}
\usepackage[T1]{fontenc}
\usepackage{natbib}
\usepackage{color}
\usepackage{amsmath, amssymb}

\newcommand{\tred}[1]{\textcolor{red}{#1}}

\title{Financial Plan}
\author{Kyunghyun Cho\thanks{Department of Information and
    Computer Science, School of Science, Aalto University,
             Finland}}


\begin{document}

\vspace{20mm}

\maketitle

\tableofcontents

\vfill

\newpage

\section{Cost Summary}

Funding I request may be broken down into (1)
\textit{research}-related and (2) \textit{living}-related
costs. The research-related cost includes budgets for equipments,
research visits and conference trips, and the living-related
cost covers moving and living costs.

The research-related cost should be given priority, if the
partial funding is decided. For research-related activities, I
request one-time cost of $11,519$ EUR (equipments and research
visits) and yearly cost of $12,500$ EUR (conference trips). If
the foundation decides to fully fund my post-doctoral research
visit to the University of Montreal, the total sum directly
related to research for two years will amount to \textbf{36,519
EUR}.

For living-related activities, the one-time cost which
corresponds to moving cost amounts to $3,878$ EUR. The health
insurance for the initial three months will be $150$ EUR.
Additionally, I request $9,000$ EUR (housing and transportation)
yearly for the duration of two years. In total, the
living-related cost amounts to \textbf{22,028 EUR}.

If the foundation decides to fully fund me, the overall funding
should amount to \textbf{58,547 EUR} for the duration of two
years. Should the foundation decide the partial funding, I would
like to emphasize that the research-related cost of $36,519$ EUR
should be given priority over the living-related cost.

In the next section, detailed descriptions of my financial
request are listed.


\section{Cost Break-Down}

\subsection{Research}

\subsubsection{Equipments}


\textbf{Laptop}: 
One important reason for choosing the Machine Learning Lab at the
University of Montreal as a post-doctoral research destination is
the availability of a large amount of computing power. At
the Machine Learning Lab I can access computing resources from
more than three distinct clusters; a nation-wide cluster,
multiple state-wide clusters and the lab-specific cluster.

Those clusters are, however, mainly focused on running a massive
number of parallel jobs, which is more suitable for
systematically exploring and searching the optimal
hyperparameters and model structures. For more preliminary
experiments on understanding how models behave and their learning
dynamics are, it is important to have a dedicated, personal
high-performance computing machine.  Furthermore, the machine
needs to be portable so that research continues without
interruption even during conference trips as well as possible
research visits to international collaborators (See the next
section for the details on research visits).

MacBook Pro from Apple is one laptop that satisfies the required
levels of both performance and portability. The laptop costs
\textbf{1649 EUR} (with memory upgrade to 16G).

\textbf{Printer:}
One of the most important activities requires to conduct the
state-of-the-art research is to read latest research papers as
much as possible. To efficiently read research papers flawlessly
it is important to have a personal, working printer at home. 

HP LaserJet Pro P1606dn is one of a few models that supports fast
printing (laser-based) and two-sided printing. The price of the
model is $209$ CAD excluding tax which amounts to \textbf{170
EUR} (including tax).



\subsubsection{Research Visits}

Statistical machine translation (the main research topic)
requires expertise of various research disciplines including
statistics, linguistics, machine learning and computational
science. In order to efficiently study and implement the novel
statistical machine translation (SMT) system, it is crucial to
collaborate with research labs internationally renowned for their
expertise in machine translation.

I plan to make a visit to the following two world-renowned
researchers during my post-doctoral years to collaborate on
implementing the novel SMT system:
\begin{enumerate}
\item Prof. Holger Schwenk: University of Maine, France
\item Prof. Hal Daume III: University of Maryland, USA
\end{enumerate}

I have become acquainted with both of them during my research
visit to the University of Montreal in 2013. The visits during my
post-doctoral years will be a continuation of the on-going
collaboration.

For each visit, I would like to request for;
\begin{itemize}
\item University of Maine, France: \textbf{5200 EUR}
\begin{itemize}
    \item Flight: 1200 EUR (inter-continental)
    \item Accommodation: 3000 EUR (three months)
    \item Other travel costs: 1000 EUR
\end{itemize}
\item University of Maryland, USA: \textbf{4500 EUR}
\begin{itemize}
    \item Flight: 500 EUR
    \item Accommodation: 3000 EUR (three months)
    \item Other travel costs: 1000 EUR
\end{itemize}
\end{itemize}


\subsubsection{Conferences}

It is crucial for me to attend top-class international
conferences regularly to keep up with the latest trend in deep
learning as well as natural language processing research.
Furthermore, those international conferences are the places where
I will present my own latest research findings to other
world-class researchers, which will help me as a researcher from
Finland gain recognition.

Each conference attendance costs approximately \textbf{2500 EUR}.
The estimate is based on the conferences I attended in 2012 and
2013. The cost for each travel may further be broken down into:
\begin{enumerate}
    \item Flight: approx. 1200 EUR (intercontinental)
    \item Transportation: approx. 100 EUR
    \item Accommodation: approx. 600 EUR (6 nights)
    \item Registration Fee: approx. 600 EUR
\end{enumerate}

There are \textbf{three} main conferences in deep learning every
year that I must attend to keep up with latest research findings
from top-class scholars and to present my work to them. They are:
\begin{enumerate}
    \item International Conference on Machine Learning (ICML)
    \item Advances in Neural Information Processing Systems (NIPS)
    \item International Conference on Learning Representation (ICLR)
    %\item International Conference on AI \& Statistics (AISTATS)
\end{enumerate}

Additionally, as I will conduct research on applying advanced
deep learning algorithms and models to natural language
processing and statistical machine translation, it is important
for me to attend at least \textbf{two} out of the following four
conferences on statistical natural language processing:
\begin{enumerate}
    \item The Annual Meeting of the Association for Computational Linguistics (ACL)
    \item The Conference of the European Chapter of the
    Association for Computational Linguistics (EACL)
    \item International Conference on Computational Linguistics (COLING)
    \item Empirical Methods in Natural Language Processing (EMNLP)
\end{enumerate}

In total, each year I will need travel budget of at least $2500
\times 5 = \mathbf{12500}$ \textbf{EUR}.

\subsection{Moving}

\subsubsection{Immigration: Work Permit Processing Fee}

It requires a work permit to enter Canada as a post-doctoral
researcher. The processing fee is \textbf{150 CAD}, and
there will be additional cost such as postal fee (A
resident in Finland must apply for Canadian visa through the
Canadian consular in London, UK). Therefore, I would like to
request \textbf{300 EUR}.

\subsubsection{Flight}

It costs approximately \textbf{1200 EUR} (Finnair, British
Airway) for a round-trip flight between Helsinki, Finland
and Montreal, QC, Canada. As it is more expensive to book an
one-way flight (at least, 1600 EUR), I will buy a round-trip
ticket. 

\subsubsection{Parcels}

As I have already lived in Finland for more than four years,
I will need to bring more than what is allowed on the flight
(often, two 23-KG bags). Those excess luggages will be sent
using the \textit{Postal Parcel International} service
provided by Posti. Each 30-KG box costs \textbf{136 EUR},
and I am planning to send \textbf{two} boxes, totaling
\textbf{272 EUR}.

\subsubsection{Temporary Accommodation}

There is high uncertainty on how well the date of my arrival and
the availability of housing will align. I expect to stay at most
\textbf{two weeks} in a temporary accommodation. Although it may
turn out to require much lower, I would like to reserve at least
\textbf{1200 EUR} for the temporary accommodation.

\subsubsection{Furniture}

As it is not economical to move the existing furnitures from
Finland to Canada, I will have to buy a set of essential
furnitures as soon as I find a new place to stay in Montreal. In
the following list, let me list the least set of these essential
furnitures (all prices are taken from IKEA Canada, excluding
        tax):
\begin{itemize}
\item Bed (Double Size): Bed Frame $229$ CAD, Bed Base $150$ CAD,
    Mattress $149$ CAD, Duvet $139$ CAD, Covers $56$ CAD, Pillow
                         $20$ CAD
\item Desk and Chair: Desk $89$ CAD, Chair $99$ CAD, Bookcase
                         $100$ CAD
\item Lighting: Work Lamps $30$ CAD $\times$ 2
\end{itemize}

In total, I request $1,091$ CAD for buying the essential
furnitures. After including tax, this amounts to approximately
\textbf{906 EUR}.


\subsection{Living}

\subsubsection{Salary}

The monthly salary will be covered by my
host, Prof. Yoshua Bengio (CAD $50000$ per year).

\subsubsection{Housing}

Monthly rental fee for a suitable studio apartment near the
University of Montreal is approximately \textbf{700 EUR}. For
each year (\textbf{12} months), I would like to request for
\textbf{8400 EUR} each year for housing.

\subsubsection{Health Insurance}

The state of Quebec in which the University of Montreal is
situated provides social plan including health care to
all legitimate residents including post-doctoral researchers.
However, the social plan starts only after the waiting period of
three months, which means that I must have insurance
covering the initial three months. 

Sun Life Assurance Company of Canada provides a monthly health
care plan (enhanced) that covers most of basic health costs including
semi-private hospitals. The monthly fee for the plan is
\textbf{63.02} CAD. Hence, in total, I request \textbf{150} EUR
for the health insurance.



\subsubsection{Public Transportation}

Montreal is the second largest city in Canada and has an
extensive network of publication transportation including buses
and subways. Due to the large size of the city, daily commute
requires using the public transportation. It cost 77 CAD per
month. Therefore, I request \textbf{55 EUR $\times$ 12 months} a
year for public transportation (in total, 660 EUR).








%\small
%\bibliographystyle{research_plan}
%\bibliography{research_plan}





\end{document}
