%% Select the article collection mode on
% See the documentation for more information about the available class options
% If you give option 'draft' or 'draft*', the draft mode is set on
\documentclass[articlecollection]{aaltoseries}
\usepackage[utf8]{inputenc}
% lipsum generates bullshit
\usepackage{lipsum}
% Uncomment and give the correct language option if you are not writing in English
%\usepackage[<language>]{babel}

% The editor of the article collection
\author{Editor}
% The title of the collection
\title{Title of the Collection}

\begin{document}

%% The abstract of the collection
% Use this command!
\draftabstract{\lipsum[1-3]}
% If you want to include another abstract for the draft in another language,
% uncomment and give the language name as the optional argument
%\draftabstract[<language>]{\lipsum[4-6]}

%% Preface
% If you write this somewhere else than in Helsinki, use the optional location.
\begin{preface}%[Optional location (if not defined, Helsinki)]
\lipsum[1-4]
\end{preface}

\vskip22pt
\noindent Examples of references to the individual articles:
\begin{enumerate}
\item ''\refart{article1}`` by \refartauth{article1} from page \pageref{article1} on\ldots
\item \refartauth{article2} also wrote ''\ref{article2}``\ldots
\item ''\ra{article3}`` by \refartauth{article3}
\item \refartauth{article5} writes in \refart[page NN]{article5} (\ra{article5}) that\ldots
\item In \ra[in Section 1]{article6}, \refartauth{article6} states that\ldots
\end{enumerate}

%% Table of contents of the whole article collection
\tableofcontents

%% The first article
\author[John Smith and John Doe]{John Smith\aff{1} and John Doe\aff{2}}
\affiliation{\aff{1}John Smith University\newline
\aff{2}John Doe University}
\title{Written directly to the document}

% The article title
\maketitle[article1]

% The article abstract
\begin{abstract}
\lipsum[1]
\end{abstract}

% The article contents
\section{Level 2 Header}
\lipsum[1-2]

\section{Level 2 Header}
\lipsum[3-4]
\subsection{Level 3 Header}
\lipsum[5]
\subsubsection{Level 4 Header}
\lipsum[6]
\subsubsection{Level 4 Header}
\lipsum[7]
\subsection{Level 3 Header}
\lipsum[8]
\section{Level 2 Header}
\lipsum[9-10]

Let's introduce references to the second article of the collection: ''\refart{article2}`` is written by \refartauth{article2}.

%% The second article
\author[John Doe and John Smith]{John Doe\aff{1}, John Smith\aff{2}}
\affiliation{\aff{1}John Doe University\newline
\aff{2}John Smith University}
\title{Written directly to the document, part 2}

% The article title, give the article a label
\maketitle[article2]

% The article abstract
\begin{abstract}
\lipsum[1]
\end{abstract}

% The article contents
\section{Level 2 Header}
\lipsum[1-2]
\section{Level 2 Header}
\lipsum[3-4]
\subsection{Level 3 Header}
\lipsum[5]
\subsubsection{Level 4 Header}
\lipsum[6]
\subsubsection{Level 4 Header}
\lipsum[7]
\subsection{Level 3 Header}
\lipsum[8]

%% The third article, a PDF article, no need to shorten the title or the author list
\addpdfarticle{dummyarticles/dummypdfarticle1.pdf}{article3}{A4-sized PDF article, bad margins}{John Doe}{}{}

%% The fourth article, a PDF article, let's shorten the title a bit
\addpdfarticle{dummyarticles/dummypdfarticle2.pdf}{article4}{Letter-sized PDF article, bad margins}{John Doe and John Smith}{Letter-sized example}{}

%% The fifth article, a PDF article
\addpdfarticle{dummyarticles/dummypdfarticle3.pdf}{article5}{Another A4-sized PDF example}{John Doe}{}{}

%% The sixth article, typeset in LaTeX, included as the source code
% The source code does not define the author(s), the author affililation(s), or the title, and it doesn't include the \maketitle command
\author{John Doe}
\affiliation{John Doe University}
\title{Article typeset in \LaTeX, included to the main document}
\maketitle[article6]
\begin{abstract}
\lipsum[1]
\end{abstract}

\section{Level 2 Header}
\lipsum[2]
\section{Level 2 Header}
\lipsum[3-4]
\subsection{Level 3 Header}
\lipsum[5]
\subsubsection{Level 4 Header}
\lipsum[6]
\subsubsection{Level 4 Header}
\lipsum[7]
\subsection{Level 3 Header}
\lipsum[8]
\section{Level 2 Header}
\lipsum[9-10]


\end{document}